\documentclass{article}
\usepackage{graphicx,caption}
\usepackage{enumitem}
\usepackage{epstopdf,subcaption}
\usepackage{psfrag}
\usepackage{amsmath,amssymb,epsf}
\usepackage{verbatim}
\usepackage[hyphens]{url}
\usepackage{amsmath}
\usepackage{color}
\usepackage{bbm}
\usepackage{listings}
\usepackage{setspace}
\usepackage{float}
\usepackage{amsmath}
\usepackage{bm}
\definecolor{Code}{rgb}{0,0,0}
\definecolor{Decorators}{rgb}{0.5,0.5,0.5}
\definecolor{Numbers}{rgb}{0.5,0,0}
\definecolor{MatchingBrackets}{rgb}{0.25,0.5,0.5}
\definecolor{Keywords}{rgb}{0,0,1}
\definecolor{self}{rgb}{0,0,0}
\definecolor{Strings}{rgb}{0,0.63,0}
\definecolor{Comments}{rgb}{0,0.63,1}
\definecolor{Backquotes}{rgb}{0,0,0}
\definecolor{Classname}{rgb}{0,0,0}
\definecolor{FunctionName}{rgb}{0,0,0}
\definecolor{Operators}{rgb}{0,0,0}
\definecolor{Background}{rgb}{0.98,0.98,0.98}
\lstdefinelanguage{Python}{
	numbers=left,
	numberstyle=\footnotesize,
	numbersep=1em,
	xleftmargin=1em,
	framextopmargin=2em,
	framexbottommargin=2em,
	showspaces=false,
	showtabs=false,
	showstringspaces=false,
	frame=l,
	tabsize=4,
	% Basic
	basicstyle=\ttfamily\footnotesize\setstretch{1},
	backgroundcolor=\color{Background},
	% Comments
	commentstyle=\color{Comments}\slshape,
	% Strings
	stringstyle=\color{Strings},
	morecomment=[s][\color{Strings}]{"""}{"""},
	morecomment=[s][\color{Strings}]{'''}{'''},
	% keywords
	morekeywords={import,from,class,def,for,while,if,is,in,elif,else,not,and,or
		,print,break,continue,return,True,False,None,access,as,,del,except,exec
		,finally,global,import,lambda,pass,print,raise,try,assert},
	keywordstyle={\color{Keywords}\bfseries},
	% additional keywords
	morekeywords={[2]@invariant},
	keywordstyle={[2]\color{Decorators}\slshape},
	emph={self},
	emphstyle={\color{self}\slshape},
	%
}

%%% Change the following flag to toggle between questions or solutions
\def\solutions{1}

\pagestyle{empty} \addtolength{\textwidth}{1.0in}
\addtolength{\textheight}{0.5in}
\addtolength{\oddsidemargin}{-0.5in}
\addtolength{\evensidemargin}{-0.5in}
\newcommand{\ruleskip}{\bigskip\hrule\bigskip}
\newcommand{\nodify}[1]{{\sc #1}}
\newcommand{\points}[1]{{\textbf{[#1 points]}}}
\newcommand{\subquestionpoints}[1]{{[#1 points]}}
\newcommand{\xsi}{x^{(i)}}
\newcommand{\zsi}{z^{(i)}}
\newenvironment{answer}{{\bf Answer:} \sf \begingroup\color{red}}{\endgroup}%

\newcommand{\bitem}{\begin{list}{$\bullet$}%
		{\setlength{\itemsep}{0pt}\setlength{\topsep}{0pt}%
			\setlength{\rightmargin}{0pt}}}
	\newcommand{\eitem}{\end{list}}

\setlength{\parindent}{0pt} \setlength{\parskip}{0.5ex}
\setlength{\unitlength}{1cm}

\renewcommand{\Re}{{\mathbb R}}
\newcommand{\R}{\mathbb{R}}
\newcommand{\what}[1]{\widehat{#1}}

\renewcommand{\comment}[1]{}
\newcommand{\mc}[1]{\mathcal{#1}}
\newcommand{\half}{\frac{1}{2}}
\newcommand{\KL}{D_{\text{KL}}}


\begin{document}
\pagestyle{myheadings} \markboth{}{CS224n Problem Set \#3}

\ifnum\solutions=1 {
	{\huge\noindent CS 224n\\
		Problem Set \#3 Solutions: Dependency Parsing}\\\\
	LA DUC CHINH
} \else {\huge
	\noindent CS 224n, Winter 2019\\
	Problem Set \#3: Dependency Parsing
} \fi

\ruleskip

{\bf Due Wednesday, Nov 14 at 11:59 pm on Gradescope.}

\medskip

\textbf{Written: Dependency Parsing}

\textbf{1. Machine Learning \& Neural Networks}

\begin{enumerate}[label=(\alph*)]
	\item Adam Optimizer
	
	\begin{align}
	\bm{\theta} \leftarrow\bm{\theta} - \alpha \nabla_{\bm{\theta}} J_{minibatch}(\bm{\theta})
	\end{align}
	
	\begin{enumerate}[label=\roman*.]
		
		\item Momentum stops updates from varying because the momentum is slowly update with $\beta_{1}$ 
		\item I don't know :(
		
	\end{enumerate}

	\item Drop out
	
	\begin{enumerate}[label=\roman*.]
		
		\item \begin{equation*}
		E_{p_{drop}}[\bm{h}_{drop}]_{i} = \gamma h_{i} (1 \times (1 - p_{drop}) + 0 \times p_{drop}) = h_{i}
		\iff \gamma h_{i}(1- p_{drop}) = h_{i}
		\iff \gamma = \frac{1}{1-p_{drop}}
		\end{equation*}
		\item Because when we want to evaluate we need our model to work at full capacity.
		
	\end{enumerate}
	
\end{enumerate}

\textbf{2. Neural Transition - Based Dependency Parsing}

\begin{enumerate}[label=(\alph*)]
	\item Transition step
	\begin{center}
	
		\begin{tabular}{p{3.5cm}|p{3.5cm}|p{3.5cm}|p{3.5cm}}
			Stack & Buffer & New dependency & Transition \\
			\hline \hline
			[ROOT] 	 & [I, parsed, this, sentence, correctly] & & Initial Configuration \\
			\hline
			[ROOT,I] & [parsed, this, sentence, correctly]    & & SHIFT \\
			\hline
			[ROOT,I,parsed] & [this, sentence, correctly] & & SHIFT \\
			\hline
			[ROOT, parsed] & [this, sentence, correctly] & parsed $\rightarrow$ I & LEFT-ARC \\
			\hline
			[ROOT, parsed, this] & [sentence, correctly] & & SHIFT \\
			\hline 
			[ROOT, parsed, this, sentence] & [correctly] & & SHIFT \\
			\hline
			[ROOT, parsed, sentence] & [correctly] & sentence $\rightarrow$ this & LEFT-ARROW \\
			\hline
			[ROOT, parsed] & [correctly] & parsed $\rightarrow$ sentence & RIGHT-ARROW \\
			\hline
			[ROOT, parsed, correctly] & [ ] & & SHIFT \\
			\hline 
			[ROOT, parsed] & [ ] & parsed $\rightarrow$ correctly & RIGHT-ARROW \\
			\hline 
			[ROOT] & [ ] & ROOT $\rightarrow$ parsed & RIGHT-ARROW
		\end{tabular}
	\end{center}

	\item n words will need $n$ SHIFT steps and $n$ step(s) of X-ARROW so in total we need $2n$ steps
\end{enumerate}

\end{document}