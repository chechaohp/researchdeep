\clearpage
\item \points{10} {\bf PCA} 

In class, we showed that PCA finds the ``variance maximizing'' directions onto
which to project the data.  In this problem, we find another interpretation of PCA. 

Suppose we are given a set of points $\{x^{(1)},\ldots,x^{(m)}\}$. Let us
assume that we have as usual preprocessed the data to have zero-mean and unit variance
in each coordinate.  For a given unit-length vector $u$, let $f_u(x)$ be the 
projection of point $x$ onto the direction given by $u$.  I.e., if 
${\cal V} = \{\alpha u : \alpha \in \Re\}$, then 
\[
f_u(x) = \arg \min_{v\in {\cal V}} ||x-v||^2.
\]
Show that the unit-length vector $u$ that minimizes the 
mean squared error between projected points and original points corresponds
to the first principal component for the data. I.e., show that
$$ \arg \min_{u:u^Tu=1} \sum_{i=1}^m \|x^{(i)}-f_u(x^{(i)})\|_2^2 \ .$$
gives the first principal component.


{\bf Remark.} If we are asked to find a $k$-dimensional subspace onto which to
project the data so as to minimize the sum of squares distance between the
original data and their projections, then we should choose the $k$-dimensional
subspace spanned by the first $k$ principal components of the data.  This problem
shows that this result holds for the case of $k=1$.

\begin{answer}
	\begin{enumerate}[label=(\alph*)]
		\item $K_{1}$ is a kernel $\rightarrow$ $z^{T}K_{1}z \geq 0$
		$K_{2}$ is a kernel $\rightarrow$ $z^{T}K_{2}z \geq 0$
		\begin{align*}
		\Rightarrow z^{T}Kz = z^{T}(K_{1}+K_{2})z = z^{T}K_{1}z+z^{T}K_{2}z \geq 0
		\end{align*}
		$\Rightarrow$ It is a kernel.
		
		\item Let $K_{1} = 0$ then we have $K=-K_{2} \Rightarrow z^{T}Kz = -(z^{T}K_{2}z) \leq 0 \Rightarrow$ Not a kernel.
		
		\item $z^{T}Kz = az^{T}K_{1}z \geq 0 \Rightarrow$ It is a kernel.
		
		\item $z^{T}Kz = -az^{T}K_{1}z \leq 0 \Rightarrow$ It is not a kernel.
		
		\item Since $K^1$ is a valid kernel, we can assume that there exists some function $\phi^1$ such that $K^1_{ij} = \phi^1(x^{(i)})^T \phi^1(x^{j})$. The same for $K^2$. In this case, 
		
		\begin{align}
		z^TK z &= \sum_{i=1}^m\sum_{j=1}^m z_i\phi^1(x^{(i)})^T\phi^1(x^{(j)})\phi^2(x^{(i)})^T\phi^2(x^{(j)})z_j\\
		&=\sum_{i=1}^m\sum_{j=1}^m z_i\sum_{k=1}^n\phi^1(x^{(i)})_k\phi^1(x^{(j)})_k\sum_{p=1}^n\phi^2(x^{(i)})_p\phi^2(x^{(j)})_pz_j\\
		&=\sum_{k=1}^n\sum_{p=1}^m (\sum_{i=1}^mz_i\phi^1(x^{(i)})_k\phi^2(x^{(i)})_p)(\sum_{j=1}^mz_i\phi^1(x^{(j)})_k\phi^2(x^{(j)})_p)\\
		&=\sum_{k=1}^n\sum_{p=1}^m (\sum_{i=1}^mz_i\phi^1(x^{(i)})_k\phi^2(x^{(i)})_p)^2 \ge 0
		\end{align}
		It is a kernel.
		
		\item Yes. This is symmetric in that $K_{ij} = K(x^{(i)}, x^{(j)}) = f(x^{(i)})f(x^{(j)}) = f(x^{(j)})f(x^{(i)}) = K_{ji}$. And
		
		$$
		z^TKz = \sum_{i,j=1}^m z_if(x^{(i)})f(x^{(j)}) z_j= (z^Tx)^2
		$$
		
		If we let $x_i = f(x^{(i)})$.
	\item $K_{3}$ is a kernel $\implies z^{T}K_{3}z \geq 0\implies z^{T}Kz = z^{T}K_{3}z \geq 0 \implies$ it is a kernel.  
	
	\item The polynomial can be written as $a_tK_3(x,z)^t+a_{t-1}K_3(x,z)^{t-1}+a_{t-2}K_3(x,z)^{t-2}+\dots+a_1K_3(x,z)+a_0$\\
	From (e) the power of a Kernel is a Kernel\\
	From (c) a positive number times a Kernel is a Kernel\\
	Therefore each of the terms of the polynomial is a Kernel\\
	Finally from (a) the sum of 2 kernels is a kernel. Therefore the polynomial is a kernel.
	\end{enumerate}
	
\end{answer}
  